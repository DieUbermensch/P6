\subsection{Laplace Transform}
From the previous sections tree expressions have been derived. One for the motor model and one for the pendulum model. Moreover an expression, which is the link between the two models, is found. 
The motor model is the following:
\begin{equation}
\tau_a(t) = \frac{k_t}{R_A} \left( V_a(t) - K_e \cdot \dot\theta_m(t) \right) - \ddot\theta_m(t)J_{T} - \dot\theta_m(t)B_{T}  \label{eq:motorfinal2}
\end{equation}
Laplace transforming \autoref{eq:motorfinal2} yields:
\begin{align}
\tau_a(s)&=\frac{k_t}{R_A}- \Theta_m(s) (s\cdot \frac{k_t+k_e}{R_a}+s\cdot B_T+s^2\cdot J_T)\\
&=\frac{k_t}{R_A}- \Theta_m(s)(s^2\cdot J_T+s(\frac{k_t+k_e}{R_a}+B_T))
\end{align}
The link between the two models shall be laplace transformed as well. The link in time domain is:
\begin{align}
 \hat{\ddot\theta}_m(t)=\frac{(J_p+m_p\cdot l^2)\hat{\ddot \theta}_p(t)}{m_p\cdot l\cdot r_w\cdot N_{ms}\cdot N_{sw}}-\frac{g\cdot \hat{\theta}_p(t)}{r_w\cdot N_{ms}\cdot N_{sw}}\label{eq:linktime}
\end{align}
Laplace transforming \autoref{eq:linktime} yields:
\begin{align}
s^2\cdot \Theta_m(s)&= \frac{(J_p+m_p\cdot l^2)s^2\cdot \Theta_p(s)}{m_p\cdot l \cdot r_w \cdot N_{ms}\cdot N_{sw}}-\frac{g\cdot \Theta_p(s)}{r_w\cdot N_{ms}\cdot N_{sw}}\\
\Rightarrow \Theta_m(s)&= \frac{(J_p+m_p\cdot l^2)\Theta_p(s)}{m_p\cdot l \cdot r_w \cdot N_{ms}\cdot N_{sw}}-\frac{g\cdot \Theta_p(s)}{s^2 \cdot r_w\cdot N_{ms}\cdot N_{sw}}
\end{align}
Lastly the model for the pendulum. The model expression is: 
\begin{equation}
(J_p+m_p\cdot l^2)\hat{\ddot \theta}_p(t)(m_p+m_c)=(m_p^2+m_c)l\cdot g \cdot \hat{\theta}_p(t)+m_p^2\cdot l^2 \cdot  \hat{\ddot\theta}_p(t)+2\cdot F_F(\theta_p(t))\label{eq:penmodel11}
\end{equation}
Laplace transforming \autoref{eq:penmodel11} yields:
\begin{equation}
s^2(J_p+m_p\cdot l^2)\Theta_p(s)(m_p+m_c)=(m_p^2+m_c)l\cdot g \cdot \Theta_p(s) + m_p^2 \cdot l^2 \cdot s^2\cdot \Theta_p(s) +2\cdot F_F(\Theta_p(s))
\end{equation}
As the input to pendulum from the motors are the applied force, it is preferable to have this linearized model as an expression of the applied force. 
\begin{equation}
2\cdot F_F(\Theta_p(s))=\Theta_p(s)(s^2((J_p+m_p\cdot l^2)(m_p+m_c)-m_p^2\cdot l^2)-(m_p^2 + m_c)l\cdot g)
\end{equation}

%
%\begin{align}
%&s^2\cdot \theta\cdot (l\cdot (m_p+m_c-m_p))-\theta \cdot g(m_p+m_c)=F_F\nonumber\\
%&\Rightarrow \frac{\theta}{F_F}=\frac{1}{s^2\cdot (l\cdot (m_p+m_c)-g(m_p+m_c)}\nonumber \\
%&\Rightarrow \frac{\frac{1}{l\cdot m_c}}{s^2-\frac{g(m_p+m_c}{l\cdot m_c}}
%\end{align}
%It shows, that the transfer function is the same as the one derived at with the taylor expansion approximation. \\
%The linear approximation of the inverted pendulum model is therefore assumed valid. 
%
%
%
%%\Rightarrow\frac{2.377}{s^2 - 37.369}
%%\Rightarrow \theta(s)\cdot s^2\cdot l ((m_p+m_c)-m_p)=g\cdot \theta(s)\cdot (m_p+m_c)+F_F(s)\nonumber\\
%%\Rightarrow \theta(s)\cdot s^2\cdot l(m_p+m_c-m_p)-g\cdot \theta(s)\cdot(m_p+m_c)=F_F(t)\nonumber\\
%%\Rightarrow \theta(s)\cdot (s^2\cdot l \cdot (m_p+m_c))=F_F(t) \nonumber \\
%The transfer function for the inverted pendulum shall be merged with a transfer function for the motors and wheels model to allow design and implementation of a controller. Merging the transfer functions is done in the following section. 
