\chapter*{Preface}\label{ch:forord}%\addcontentsline{toc}{chapter}{Forord}
This report has been carried out during spring of 2016 as a 6. semester Electronics and IT student bachelor project at Aalborg University by group 16grXXX. The project concerns the development of Somehting extremely awesome.
\\\\
%The report has the following structure: First, the segway platform is described and requirements to the controller are made. Hereafter, a model of the system is derived, and the controller can be made. Also, two digital filters are designed, together with a communication protocol between the segway and the wireless remote controller. Finally, an acceptance test and a discussion is made.\\
%The reader of the report is assumed to have a basic understanding of physics, modelling electromechanical systems and basic controller design.
%The reader is also assumed having an understanding of electronic components, C-code, filters and methods for filter design, as well as a knowledge of the frequency domain.
%References to sources are of the APA-style, i.e. of the type [\emph{source name/author's surname, year, optional page number}]. Sources are listed in a bibliography at the end of the report, and PDFs, datasheets etc. referred to throughout the report can be found on the attached CD. Figures without a source are made by the project group.% Appendices are listed after the bibliography.

%This project has been supervised by assistant professor Christoffer Eg Sloth as main supervisor with PhD fellow Rasmus Pedersen as co-supervisor. 
%The project group would like to thank Aalborg University for providing the segway in the project. 
%Also thanked is engineering assistant at Institute for Electronics Systems at AAU, Simon Jensen, for his help with the hardware used in the project. Machine worker Jesper Dejgaard Pedersen at Aalborg University is also thanked, for his help with the wheels and gears on the segway. Finally, group 15gr633 are thanked for their help with discussing the workings of the system, and providing the framework for the \iic driver functions.\\
\vspace{0.5\baselineskip}\hfill Aalborg University, May 27, 2016
%\vfill

%% underskrifts afsnit %%
\hspace{1.5\baselineskip}
\makebox[\textwidth][c]{
\begin{minipage}[c]{0.45\textwidth}
 \centering
 \rule{\textwidth}{0.45pt}\\
  Poul Hoang\\
 {\footnotesize phoang13@student.aau.dk}
\end{minipage}}
\hspace{1.5\baselineskip}

\vspace{1.5\baselineskip}

\begin{minipage}[b]{0.45\textwidth}
 \centering
 \rule{\textwidth}{0.45pt}\\
  Mikkel Krogh Simonsen\\
 {\footnotesize mksi13@student.aau.dk}
\end{minipage}
\vspace{1.5\baselineskip}
\hfill
\begin{minipage}[b]{0.45\textwidth}
 \centering
 \rule{\textwidth}{0.45pt}\\
  Kasper Kiis Jensen\\
 {\footnotesize rsader13@student.aau.dk}
\end{minipage}

