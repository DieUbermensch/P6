%pagestyle{fancy} %enable headers and footers again

\begin{comment}
\pdfbookmark[0]{Danish title page}{label:titlepage_en}
\aautitlepage{%
  \englishprojectinfo{
    Project Title %title
  }{%
    Analoge kredsløb og systemer %theme
  }{%
    P3: 2. September 2014 - 17. December 2014 %project period
  }{%
    14gr313 % project group
  }{%
    %list of group members
    Amalie Vistoft Petersen\\
    Mikkel Krogh Simonsen\\
    Rasmus Gundorff Sæderup\\
    Simon Bjerre Krogh\\
    Thomas Kær Juel Jørgensen\\
    Thomas 'Godlike' Rasmussen
  }{%
    %list of supervisors
    Tom S. Pedersen

  }{%
    9 % number of printed copies
  }{%
    \today % date of completion
  }%
}{%department and address
  \textbf{Institut for Elektroniske Systemer}\\
  Fredrik Bajers Vej 7\\
  DK-9220 Aalborg Ø\\
  }{% the abstract
  Here is the abstract
}

\cleardoublepage

\end{comment}

\selectlanguage{english}
\pdfbookmark[0]{Titelblad}{label:titelblad}
\aautitlepage{%
  \danishprojectinfo{
    Segway%title
  }{%
    Digital and Analog Systems Interacting with the Surroundings  %theme
  }{%
    5. Semester %project period
  }{%
    15gr514 % project group
  }{%
    %list of group members
    Andrea Victoria Tram Løvemærke\\
    Poul Hoang\\
    Ralf Victor Lømand Ravgård Christiansen\\
    Rasmus Gundorff Sæderup\\
    Thomas Paul Guyot\\
    Thomas Kær Juel Jørgensen
  }{%
    %list of supervisors
    Christoffer Sloth
    }{
    Rasmus Pedersen
    }{%
    9 % number of printed copies
  }
  {%
    December 17, 2015%\today % date of completion
  }%
}{%department and address
  \textrm{\textbf{Institute of Electronic Systems  }\\
  Fredrik Bajers Vej 7\\
  DK-9220 Aalborg Ø\\}
 }{
 In this project, a controller has been designed to make a segway stand in an upright position. A model of the segway has been derived, describing it as two motors moving the base of an inverted pendulum. Tests show a fit of 84\% for the motor model and 70\% for the inverted pendulum model. Based on these models, two controllers are designed. A P-controller for the motor and a PID-controller for the inverted pendulum, coupled in a cascade controller.
 
The controller is able to stabilize the model of the system, and fulfilling the requirements for the settling time and rise time, but with an overshoot of $23.5 \%$ and a steady-state error of $12.3 \%$.
Digital filters are designed to reduce the noise in the sensor measurements. The filters are low-pass Butterworth filters, which are transformed into the digital domain using bilinear transformation, but they are not implemented due to computational power constraints.
A communication protocol is implemented, in which a package structure and package header has been designed, to facilitate the wireless communication between the segway and a remote controller.

After tuning the controller parameters, the segway is able to stand upright, and is also able to withstand a light push. It is also possible to drive the segway back and forth using the remote controller, as well as receive data from the segway wirelessly.
}