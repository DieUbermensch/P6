\begin{tikzpicture}[auto, node distance=3.5cm,>=latex']
    % We start by placing the blocks
    \node [input, name=input] {};

    % Once the nodes are placed, connecting them is easy. 
	\draw[thick,dashed, fill=black!10, align=center] ($(input.north west)+(15.5,-1)$) rectangle ($(input.north west)+(0.6,2.75)$);
	\draw[thick,dashed, fill=black!30, align=center] ($(input.north west)+(10.75,-0.8)$) rectangle ($(input.north west)+(1.5,1.75)$);
	 \node [align=center] at ($(input.north west)+(8,2.25)$) {\textbf{Plant, }$\mathbf{ \, \, \, G}$};
	 \node [align=center] at ($(input.north west)+(6,1.25)$) {{Motors and Wheels Model}};
	
	\node [blockbig, right of=input, align = center] (motor) {Motor \\ and Wheel};
	\node [blockbig, right of=motor, align = center, node distance=4.5cm] (Cart) {Cart \\ Movement};
    \node [blockbig, right of=Cart, align = center, node distance=4.5cm] (pendulum) {Inverted \\ Pendulum};

   \node [output, right of=pendulum, node distance=3.5cm] (output) {};

	
    \draw [->] (input) -- node {$V_a$} (motor);
    \draw [->] (4.75,0.2) -- node {$\tau_a$} (6.75,0.2);
    \draw [->] (9.25,0.2) -- node {$\theta_w$} (11.25,0.2);
    \draw [->] (11.25,-0.2) -- node {$F_L$} (9.25,-0.2);
   % \draw [->] (9.75,-0.2) -| node {} (3.75,-0.2);
    \draw [->] (pendulum) -- node {$\theta_p$} (output);
\end{tikzpicture}