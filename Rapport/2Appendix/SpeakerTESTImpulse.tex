\chapter{Speaker impulse Lab tests} \label{app:journal_speaker_test}

The purpose of this test is to examine how a short applied impulse effects the speaker at different sound levels, up until the coil hits the backplate. The thesis of the test is that the loudspeaker will act as an bandpass filter when the coil does not hit the backplate because the loudspeaker is not able to play sounds between 0 - 20 Hz. but if the coil does hit the backplate the loudspeaker will only act as a lowpass filter.

%The distortion level is to be viewable by spectral estimation of accelerometers placed inside the speaker enclosure. 

%It is desirable to find the situation where the woofer hits its coil, since the is the least wanted situation of a playing speaker. The distortion or vibration measured from the speaker can later be used to regulate the input and prevent further distortion to appear. The purpose is expressly formulated as:

%\begin{itemize}
%\item Determine the vibration frequency response of the enclosure and driver when applying a short impulse.
%\item Determine the vibration frequency response of the enclosure and driver when applying a short impulse and the membrane hits the coil.
%\end{itemize}

\section{Setup}

The setup of this experiment are depicted in Figure \ref{figure:SpeakertestSetup}, where the equipment is catalogued in \autoref{tab:UsedEquipment1}, and described as follows:

\todo[inline]{Skriv vores egen accelerometer setup ind.}
\begin{itemize}
\item Distortion and \gls{SPL} will be measured by a microphone at a distance 1 meter in accordance with IEC 60268-5 Sound System Equipment - Part 5: Loudspeaker.
\item Vibration will be measured by a Brüel \& Kjear Type 4344 accelerometer, placed at:
\begin{itemize}
\item The backplate of the lowest woofer
\item High, inside and on the back of the enclosure 
\end{itemize}
\item Vibration will also be measured by a ADXL335 accelerometer, placed in the same area (maximum 3 cm away) as the Brüel \& Kjear accelerometers. 
\item The speaker will be driven by a Crown Studio Reference I amplifier.
\item The ADC/DAC will convert measurements from accelerometers and microphone and relay to a computer via SPDIF.
\begin{itemize}
\item Both Accelerometers and Microphone is calibrated into outputting -37 dB at respectively 1 G and 94 dB \gls{SPL}.
\item All recordings are synchronised and timestamped with by looping the test file back into the converter.
\end{itemize}
\item The computer will be logging data with a RME HammerFall DIGI 96-PDST sound card and Adobe Audition.
\end{itemize}

Furthermore the speaker will be placed in the anechoic room to eliminate any external disturbances corresponding to the requirements demanded in the 
IEC 60268-5 standard.

\subsection*{Test Setup}
\todo[inline]{Fiks billede i forhold til test}

\begin{figure}[H]
\centering
\tikzsetnextfilename{TestSetup1}
\scalebox{0.7}{
\begin{tikzpicture}
\node[input] (SpeakerCorner) at (0,0) {};
\node[PreAmpBox] (AMP0) at ($(6,-1.5)+(SpeakerCorner)$) {\scalebox{0.5}{Pre-amp}} ;
\node[PreAmpBox] (AMP1) at ($(0,-1)+(AMP0)$) {\scalebox{0.5}{Pre-amp}} ;
\node[PreAmpBox] (AMP2) at ($(0,-1)+(AMP1)$) {\scalebox{0.5}{Pre-amp}} ;
%\node[PreAmpBox] (AMP3) at ($(0,-1)+(AMP2)$) {\scalebox{0.5}{Pre-amp}} ;
%\node[PreAmpBox] (AMP4) at ($(0,-1)+(AMP3)$) {\scalebox{0.5}{Pre-amp}} ;
%\node[PreAmpBox] (AMP5) at ($(0,-1)+(AMP4)$) {\scalebox{0.5}{Pre-amp}} ;


%% Speaker Enclosure %%
%% Speaker %%
\draw[thick, fill=black!20] (SpeakerCorner) -- ($(2,0)+(SpeakerCorner)$) -- ($(2,-6)+(SpeakerCorner)$) -- ($(0,-6)+(SpeakerCorner)$) -- (SpeakerCorner);

\node[input] (Tweeter) at ($(0,-0.5)+(SpeakerCorner)$) {};
\draw[thick, fill=black!20] (Tweeter) -- ($(0,0.3)+(Tweeter)$) -- ($(0.5,0.25)+(Tweeter)$) -- ($(0.5,-0.25)+(Tweeter)$) -- ($(0,-0.3)+(Tweeter)$) -- (Tweeter);
\draw[thick, fill=black!20] ($(0.51,-0.25)+(Tweeter)$) -- ($(0.75,-0.25)+(Tweeter)$) -- ($(0.75,0.25)+(Tweeter)$) -- ($(0.51,0.25)+(Tweeter)$);

\node[input] (Speaker2) at ($(0,-2)+(SpeakerCorner)$) {};
\draw[thick, fill=black!20] (Speaker2) -- ($(0,0.5)+(Speaker2)$) -- ($(0.5,0.25)+(Speaker2)$) -- ($(0.5,-0.25)+(Speaker2)$) -- ($(0,-0.5)+(Speaker2)$) -- (Speaker2);
\draw[thick, fill=black!20] ($(0.51,-0.25)+(Speaker2)$) -- ($(0.75,-0.25)+(Speaker2)$) -- ($(0.75,0.25)+(Speaker2)$) -- ($(0.51,0.25)+(Speaker2)$);

\node[input] (Speaker) at ($(0,-3.5)+(SpeakerCorner)$) {};
\draw[thick, fill=black!20] (Speaker) -- ($(0,0.5)+(Speaker)$) -- ($(0.5,0.25)+(Speaker)$) -- ($(0.5,-0.25)+(Speaker)$) -- ($(0,-0.5)+(Speaker)$) -- (Speaker);
\draw[thick, fill=black!20] ($(0.51,-0.25)+(Speaker)$) -- ($(0.75,-0.25)+(Speaker)$) -- ($(0.75,0.25)+(Speaker)$) -- ($(0.51,0.25)+(Speaker)$);

\draw[thick, fill=black!20] ($(2,-5.5)+(SpeakerCorner)$) -- ($(1,-5.5)+(SpeakerCorner)$) -- ($(1,-5)+(SpeakerCorner)$) -- ($(2,-5)+(SpeakerCorner)$) -- ($(2,-5.5)+(SpeakerCorner)$);

%% Computer + AMP %%
\begin{pgfonlayer}{bg}
\node[input] (ComputerCorner) at (8,-1) {};
\draw[thick, fill=black!20] (ComputerCorner) -- ($(2,0)+(ComputerCorner)$) -- ($(2,-4)+(ComputerCorner)$) -- ($(0,-4)+(ComputerCorner)$) -- (ComputerCorner);
\end{pgfonlayer}{bg}

\node[] (Computer) at ($(1,-2)+(ComputerCorner)$) {ADC/DAC};

\node[PreAmpBox] (AMP) at ($(1,-5)+(ComputerCorner)$) {\scalebox{0.5}{Amplifier}} ;

%% Draw Lines %%

%% Pre-amp to Speaker
\draw[<-,mark=*] (AMP0) -- ($(-3,0)+(AMP0)$) |- ($(-1.5,0.5)+(SpeakerCorner)$) -- ($(-1.5,-2)+(SpeakerCorner)$) ;
\draw [red,fill] ($(-1.5,-2)+(SpeakerCorner)$) circle [radius=0.1];

\draw[<-,mark=*] (AMP1) -| ($(-4.25,1.5)+(AMP1)$) ;
\draw [blue,fill] ($(-4.25,1.5)+(AMP1)$) circle [radius=0.1];

\draw[<-,mark=*] (AMP2) -- ($(-5.25,-0)+(AMP2)$) ;
\draw [blue,fill] ($(-5.25,-0)+(AMP2)$) circle [radius=0.1];

%\draw[<-,mark=*] (AMP3) -- ($(-5.5,0)+(AMP3)$) ;
%\draw [blue,fill] ($(-5.5,0)+(AMP3)$) circle [radius=0.1];

%\draw[<-,mark=*] (AMP4) -- ($(-4.5,0)+(AMP4)$) ;
%\draw [blue,fill] ($(-4.5,0)+(AMP4)$) circle [radius=0.1];

%\draw[<-,mark=*] (AMP5) -| ($(-3.5,0.25)+(AMP5)$) -| ($(1.5,-5.25)+(SpeakerCorner)$) ;
%\draw [blue,fill] ($(1.5,-5.25)+(SpeakerCorner)$) circle [radius=0.1];

\draw[->,mark=*] (AMP) -| ($(2.25,-5.75)+(SpeakerCorner)$) |- ($(2,-5.4)+(SpeakerCorner)$) ;

%% Amps to Computer

\draw[<-,mark=*] (AMP) -- ($(1,-4)+(ComputerCorner)$) ;
\draw[->] (AMP0) -| ($(-0.5,-0.5)+(ComputerCorner)$) -- ($(0,-0.5)+(ComputerCorner)$) ;
\draw[->] (AMP1) -- ($(0,-1.5)+(ComputerCorner)$) ;
\draw[->] (AMP2) -- ($(0,-2.5)+(ComputerCorner)$) ;
%\draw[->] (AMP3) -| ($(-0.75,-2)+(ComputerCorner)$)  -- ($(0,-2)+(ComputerCorner)$) ;
%\draw[->] (AMP4) -| ($(-0.75,-3)+(ComputerCorner)$)  -- ($(0,-3)+(ComputerCorner)$) ;
%\draw[->] (AMP5) -| ($(-0.75,-3.75)+(ComputerCorner)$)  -- ($(0,-3.75)+(ComputerCorner)$) ;
%\draw[->] (AMP5) -| ($(-0.75,-3.75)+(ComputerCorner)$)  -- ($(0,-3.75)+(ComputerCorner)$) ;
%\draw[->] (AMP5) -| ($(-0.75,-3.75)+(ComputerCorner)$)  -- ($(0,-3.75)+(ComputerCorner)$) ;
%% Rooms

\draw[->] ($(0,0.75)+(AMP)$) -- ($(-2,0.75)+(AMP)$) -- ($(-2,1.25)+(AMP)$) -- ($(-1,1.25)+(AMP)$);

\draw[thick, dashed] ($(-2,1)+(SpeakerCorner)$) -- ($(10.5,1)+(SpeakerCorner)$) -- ($(10.5,-7)+(SpeakerCorner)$) -- ($(-2,-7)+(SpeakerCorner)$) -- ($(-2,1)+(SpeakerCorner)$);
\draw node [black, above=0.05] at ($(-0.25,-7)+(SpeakerCorner)$) {Anechoic room};


%% To Computer %%
\begin{pgfonlayer}{bg}
\node[input] (PC) at (13,-1) {};
\draw[thick, fill=black!20] (PC) -- ($(2,0)+(PC)$) -- ($(2,-2)+(PC)$) -- ($(0,-2)+(PC)$) -- (PC);
\end{pgfonlayer}{bg}


\draw[<->] ($(-3,-1)+(PC)$) -- node[above] {SPDIF} ($(0,-1)+(PC)$) ;
\node[] (PC1) at ($(1,-1)+(PC)$) {Computer};

\end{tikzpicture}
}
\caption{Test setup}
\label{figure:SpeakertestSetup2}
\end{figure}

\subsection*{Equipment used and AAU-no.}

\begin{table}[H]
\centering
\ra{1.3}
\begin{tabular}{S[table-format=1]ccc} \toprule
    {Item} & {Description} & {AAU-no} \\ \bottomrule 
    1      &  B \& K Accelerometer Type 4344  & 64659   \\ 
    2      &  B \& K Accelerometer Type 4344  & 64660   \\ 
    3      &  B \& K 2-channel Accelerometer Pre-amp Type 2622  & 07013   \\
    4      &  B \& K Microphone Type 4165  & 08132   \\
    5      &  Gras - 26AK Pre-amp & 52665   \\
    6      &  B \& K Microphone Power supply Type 2804  & 07304   \\
    7      &  Crown Studio Reference I Amplifier & 52614   \\
    8      &  BEHRINGER digital A/D \& D/A Converter - Model ADA8000   & 56545   \\
    9      &  B \& K Accelerometer calibrator 4294 & 08023   \\
    10     &  B \& K Microphone calibrator 4294 & 78301   \\
    11     &  RME HammerFall DIGI 96-PDST sound card & 60919  \\
    12     &  TBD Power Supply & TBD  \\
    13     &  ADXL335 Accelerometer (Sparkfun Breakout) & NaN  \\
    14     &  Passive Dali Zensor 5 AX & NaN  \\ \bottomrule 
\end{tabular}
\caption{Table over equipment used in test}
\label{tab:UsedEquipment2}
\end{table}



\section{Procedure}\label{sec:SpeakerTestProcedure}

The producer for this experiment is described as follows:
\begin{enumerate}
\item Adjust volume on the amplifier to +3 dB gain.
\item Vacate the anechoic room and seal the room.
\item Start recording in Adobe Audition for:
\begin{itemize}
\item Driver and enclosure accelerometers
\item Microphone and playback loop
\end{itemize}
\item Playback file \path{ImpulseX.wav}, which is an impulse. The amplitude is 0 dBFS.
\item After playback, stop recording all channels
\item Save recordings as \path{.wav} file
\item Enter room and adjust amplifier by +1 dB
\item Repeat step 2 through 7 until the coil hits the back plate.
\end{enumerate}

The \path{SingleToneX.wav} can be found on:\\
\scalebox{0.7}{
\path{CD://Maalinger/XX/Impulse.wav}}\\

\section{Data Extraction}

The recordings can be found on:\\
\scalebox{0.7}{
\path{CD://Maalinger/XX}}
And is indexed in folders \scalebox{0.8}{\path{Measure_X}}, where X corresponds to the test number. Every measurements is denoted as:
\begin{itemize}
\item Accelerometer on driver: \scalebox{0.8}{\path{Acc driver_0XX}}
\item Accelerometer in enclosure: \scalebox{0.8}{\path{Acc enclosure_0XX}}
\item ADXL335 on driver: \scalebox{0.8}{\path{ADXL335 Driver_0XX}}
\item ADXL335 in enclosure: \scalebox{0.8}{\path{ADXL335 Enclosure_0XX}}
\item Microphone: \scalebox{0.8}{\path{Mic_0XX}}
\item Playback loop: \scalebox{0.8}{\path{Reference_0XX}}
\end{itemize}

\section{Analysis}

Comments and potential further manipulation is documented in this section. 

\section{Error sources}

Potential Error sources are noted in the section

