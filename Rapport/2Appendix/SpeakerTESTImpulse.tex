\chapter{Speaker impulse Lab tests} \label{app:journal_speaker_test}

The purpose of this test is to examine how a short applied impulse effects the speakers at different sound levels, up until the membrane hits the coil. The thesis of the test is that the loudspeaker will act as an bandpass filter when the membrane does not hit the coil because the loudspeaker is not able to play sounds between 0 - 20 Hz. but if the membrane does hit the coil the loudspeaker will only act as a lowpass filter. The scope of this test is therefore:

%The distortion level is to be viewable by spectral estimation of accelerometers placed inside the speaker enclosure. 

%It is desirable to find the situation where the woofer hits its coil, since the is the least wanted situation of a playing speaker. The distortion or vibration measured from the speaker can later be used to regulate the input and prevent further distortion to appear. The purpose is expressly formulated as:

\begin{itemize}
\item Determine the vibration frequency response of the enclosure and driver when applying a short impulse.
\item Determine the vibration frequency response of the enclosure and driver when applying a short impulse and the membrane hits the coil.
\end{itemize}

\subsection{Setup}

The setup of this experiment are depicted in Figure \ref{figure:SpeakertestSetup} and described as follows:
\begin{itemize}
\item Distortion will be measured by a microphone at a distance 1 meter
\item Vibration will be measured by a Brüel \& Kjear Type \textbf{TBD}, placed at:
	\begin{itemize}
	\item The lower woofer
	\item Centrally inside the enclosure 
	\end{itemize}
\item Vibration will also be measured by a \textbf{TBD}
	\begin{itemize}
	\item The lower woofer
	\item Centrally inside the enclosure 
	\end{itemize}
\item The speaker will be driven by a \textbf{TBD} amplifier.
\item A computer will be logging data with a \textbf{TBD} soundcard
\end{itemize}

Furthermore the speaker will be placed in the anechoic room to eliminate any external disturbances.

\begin{figure}[H]
\centering
\missingfigure{Picture of setup}
\caption{test setup}
\label{figure:SpeakertestSetup}
\end{figure}

\subsection*{Equipment used and AAU-no.}

\begin{table}[H]
\centering
\ra{1.3}
\begin{tabular}{S[table-format=1]ccc} \toprule
    {Item} & {Description} & {AAU-no} \\ \bottomrule 
    1      &  B\& K Accelerometer Type 4344  & 64659   \\ 
    2      &  B\& K Accelerometer Type 4344  & 64660   \\ 
    3      & YY  & XX   \\
    4      & YY  & XX   \\ 
    5      & YY  & XX  \\ \bottomrule 
\end{tabular}
\caption{Table over used equipment}
\end{table}



\section{Procedure}\label{sec:SpeakerTestProcedure}

The producer for this experiment is described as follows:
\begin{enumerate}
\item Adjust volume on the amplifier.
\item Playback the impulse \textbf{waw.file} 
\item If the speaker could handle it, increase the volume with 1 dB.
\item Repeat until the woofer hits coil. 
\end{enumerate}

All of the accelerometers have beforehand been calibrated to have the same sensitivity.

\section{Data Extraction}

\section{Analysis}

Comments and potential further manipulation is documented in this section. 

\section{Error sources}

Potential Error sources are noted in the section

