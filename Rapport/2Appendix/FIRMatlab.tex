\chapter{Matlab script for designing FIR filters} \label{app:FIR_Matlab}
This appendix will explain the Matlab script used for designed quantisized FIR coefficients from the specifications of the standard IEC 6964 (2001) 

The matlab script is called MainDesign2 and uses the follwing self implemented functions
\begin{itemize}
\item StandardFunction2 (\autoref{ls:StandardFunctions}).
\item FIRFunctionls (\autoref{ls:matlabFirFunction}).
\item QuantizingFunction (\autoref{ls:quan}).
\end{itemize}
MainDesign2, see \autoref{ls:FIRMatlabScript} is the main script which designs 8 FIR filters using firstly the function StandardFunction2 to calculate the requirements for all eight FIR filters using the standard IEC 6964 (2001) which is explained in detail in \autoref{app:IEC6964}.

Nextly all the coefficients for the eight FIR filters are derived using the function FIRFunction, for every filter, which uses the Kaiser window method explained in \autoref{app:FIR_theory}. In The FIRFunction the function QuantizingFunction is used to quantisize all coefficients to Q15 format used by the DSP. The script and functions can be seen below. 

\begin{lstlisting}[language=Matlab, caption = {Matlab script for designing FIR filters},label={ls:FIRMatlabScript}]
close all
clear all 
clc

%% Script for designing the filters
% The script will create the filters for decimation, analyzer spectrum and
% interpolation using the functions FIRFunction, QuantizingFunction and
% StandardFunction. All the filter are made by FIR filters using the 
%kaiser window method. Made by group 640 EIT



%% Main specifications 
Fs = 48000;

%% Decimation
%Decimation is done by lowpass filtering and then downsampling the signal
%Filters are designed using the standards

%Standards
fr = 3000; %start frequency
G  = 2;  %base
b  = 1;  %bands in base
class = 2 %class of demands

% Find the standards
[f_center, alpha,G_freq] = StandardFunction2(fr,G,b,class);

figure;
semilogx(f_center(1)*G_freq,-alpha,'color',[0.7 0 0]); grid on;
hold on 
semilogx(f_center(2)*G_freq,-alpha,'color',[0 0.7 0]); grid on;
semilogx(f_center(3)*G_freq,-alpha,'color',[0 0 0.7]); grid on;
semilogx(f_center(4)*G_freq,-alpha,'color',[0.7 0.7 0]); grid on;
semilogx(f_center(5)*G_freq,-alpha,'color',[0.7 0 0.7]); grid on;
semilogx(f_center(6)*G_freq,-alpha,'color',[0 0.7 0.7]); grid on;
semilogx(f_center(7)*G_freq,-alpha,'color',[0.7 0.2 0.8]); grid on;
semilogx(f_center(8)*G_freq,-alpha,'color',[0.3 0.3 0.3]); grid on;

%% Filter specs 
% The values have been fitted so the standards are upheld 
f_pass = 3000;     % Pass band frequency 
a_pass = -1;       % Attenuation at pass band 
f_stop = 6500;    % Stop band frequency
a_stop = -60;      % Attenuation at stop band

%% Filter specs1
f_pass1 = f_pass;
a_pass1 = a_pass;
f_stop1 = f_stop;
a_stop1 = a_stop;

% Run function to obtain quatisized b coeff for filter
[bQDec1, MDec1, betaDec1] = FIRFunction(f_pass1,a_pass1,f_stop1,a_stop1,Fs); 
%figure with filter and standards to see if the filter uphelds the standard
figure;
freqz(bQDec1,1,Fs,Fs)
ax = findall(gcf, 'Type', 'axes');
set(ax, 'XScale', 'log');
hold on 
semilogx(f_center(1)*G_freq,-alpha,'r'); grid on;
%% Filter specs2
% FIlter specs and sampling frequency are halved so the same b coeff are 
% reached, figure is also made to confirm. Fs and specs continue to be 
% halved with every new filter so no more comments are made
K = 2;
f_pass2 = f_pass/K;
a_pass2 = a_pass;
f_stop2 = f_stop/K;
a_stop2 = a_stop;

[bQDec2, MDec2, betaDec2] = FIRFunction(f_pass2,a_pass2,f_stop2,a_stop2,Fs/K);
%figure;
%freqz(bQDec2,1,Fs/K,Fs/K)
%ax = findall(gcf, 'Type', 'axes');
%set(ax, 'XScale', 'log');
%hold on 
%semilogx(f_center(2)*G_freq,-alpha); grid on;
%% Filter specs3
K = 4;
f_pass3 = f_pass/K;
a_pass3 = a_pass;
f_stop3 = f_stop/K;
a_stop3 = a_stop;

[bQDec3, MDec3, betaDec3] = FIRFunction(f_pass3,a_pass3,f_stop3,a_stop3,Fs/K);
%figure;
%freqz(bQDec3,1,Fs/K,Fs/K)
%ax = findall(gcf, 'Type', 'axes');
%set(ax, 'XScale', 'log');
%hold on 
%semilogx(f_center(3)*G_freq,-alpha); grid on;
%% Filter specs4
K = 8;
f_pass4 = f_pass/K;
a_pass4 = a_pass;
f_stop4 = f_stop/K;
a_stop4 = a_stop;

[bQDec4, MDec4, betaDec4] = FIRFunction(f_pass4,a_pass4,f_stop4,a_stop4,round(Fs/K));
%figure;
%freqz(bQDec4,1,round(Fs/K),round(Fs/K))
%ax = findall(gcf, 'Type', 'axes');
%set(ax, 'XScale', 'log');
%hold on 
%semilogx(f_center(4)*G_freq,-alpha); grid on;
%% Filter specs5
K = 16;
f_pass5 = f_pass/K;
a_pass5 = a_pass;
f_stop5 = f_stop/K;
a_stop5 = a_stop;

[bQDec5, MDec5, betaDec5] = FIRFunction(f_pass5,a_pass5,f_stop5,a_stop5,round(Fs/K));
%figure;
%freqz(bQDec5,1,round(Fs/K),round(Fs/K))
%ax = findall(gcf, 'Type', 'axes');
%set(ax, 'XScale', 'log');
%hold on 
%semilogx(f_center(5)*G_freq,-alpha); grid on;
%% Filter specs6
K = 32;
f_pass6 = f_pass/K;
a_pass6 = a_pass;
f_stop6 = f_stop/K;
a_stop6 = a_stop;

[bQDec6, MDec6, betaDec6] = FIRFunction(f_pass6,a_pass6,f_stop6,a_stop6,round(Fs/K));
%figure;
%freqz(bQDec6,1,round(Fs/K),round(Fs/K))
%ax = findall(gcf, 'Type', 'axes');
%set(ax, 'XScale', 'log');
%hold on 
%semilogx(f_center(6)*G_freq,-alpha); grid on;
%% Filter specs7
K = 64;
f_pass7 = f_pass/K;
a_pass7 = a_pass;
f_stop7 = f_stop/K;
a_stop7 = a_stop;

[bQDec7, MDec7, betaDec7] = FIRFunction(f_pass7,a_pass7,f_stop7,a_stop7,round(Fs/K));
%figure;
%freqz(bQDec7,1,round(Fs/K),round(Fs/K))
%ax = findall(gcf, 'Type', 'axes');
%set(ax, 'XScale', 'log');
%hold on 
%semilogx(f_center(7)*G_freq,-alpha); grid on;

%% Upsamling
f_passInt = 9600;
a_passInt = -1;
f_stopInt = 16800;
a_stopInt = -80;

% Run function to obtain quatisized b coeff for filter
[bQInt1, MInt1, betaInt1] = FIRFunction(f_passInt,a_passInt,f_stopInt,a_stopInt,Fs); 
%figure with filter and standards to see if the filter uphelds the standard
figure;
freqz(bQInt1,1)
ax = findall(gcf, 'Type', 'axes');
set(ax, 'XScale', 'log');
hold on 
semilogx(f_center(1)*G_freq,-alpha,'r'); grid on;

%% Delays2
delay1  = (MInt1/2) + 2*(MInt1/2)+25*2 + 4*(MInt1/2)+4*25 + 8*(MInt1/2)+25*8 + 16*(MInt1/2)+25*16 + 32*(MInt1/2)+25*32 + 64*(MInt1/2)+25*64
delay2  = (MInt1/2) + 2*(MInt1/2)+25*2 + 4*(MInt1/2)+4*25 + 8*(MInt1/2)+25*8 + 16*(MInt1/2)+25*16 + 32*(MInt1/2)+25*32
delay4  = (MInt1/2) + 2*(MInt1/2)+25*2 + 4*(MInt1/2)+4*25 + 8*(MInt1/2)+25*8 + 16*(MInt1/2)+25*16
delay8  = (MInt1/2) + 2*(MInt1/2)+25*2 + 4*(MInt1/2)+4*25 + 8*(MInt1/2)+25*8
delay16 = (MInt1/2) + 2*(MInt1/2)+25*2 + 4*(MInt1/2)+4*25
delay32 = (MInt1/2) + 2*(MInt1/2)+25*2
delay64 = (MInt1/2)

\end{lstlisting}

\begin{lstlisting}[language=Matlab, caption = {Matlab script for standard functions},label={ls:StandardFunctions}]
function [fm, alpha, G_freq] = StandardFunction2(fr,G,b,class)
%% Script for the calculations of filters from standard IEC-61260 (1995)
%% frequency 
%Using the octave band (base two G2 = 2) and not the base ten 
for x = 0:13                % Get center frequencies for 14 bands (only need 8)
fm(x+1) = (G^(-x/b))*fr;    % Calculate the center frequencies from the input 
end

% Attenuation at higher frequency than fm
% Calculate some points where the attenuations also will be calculated
G_freq(10)=1+((G^(1/(2*b))-1)/(G^(1/2)-1))*(G^(0)-1);
G_freq(11)=1+((G^(1/(2*b))-1)/(G^(1/2)-1))*(G^(1/8)-1);
G_freq(12)=1+((G^(1/(2*b))-1)/(G^(1/2)-1))*(G^(1/4)-1); 
G_freq(13)=1+((G^(1/(2*b))-1)/(G^(1/2)-1))*(G^(3/8)-1); 
G_freq(14)=1+((G^(1/(2*b))-1)/(G^(1/2)-1))*(G^(1/2)-1); 
G_freq(15)=1+((G^(1/(2*b))-1)/(G^(1/2)-1))*(G^(1/2)-1); 
G_freq(16)=1+((G^(1/(2*b))-1)/(G^(1/2)-1))*(G^(1)-1); 
G_freq(17)=1+((G^(1/(2*b))-1)/(G^(1/2)-1))*(G^(2)-1); 
G_freq(18)=1+((G^(1/(2*b))-1)/(G^(1/2)-1))*(G^(3)-1); 
G_freq(19)=1+((G^(1/(2*b))-1)/(G^(1/2)-1))*(G^(4)-1); 

%inverted frequencies
G_freq(1)=1/G_freq(19);
G_freq(2)=1/G_freq(18);
G_freq(3)=1/G_freq(17);
G_freq(4)=1/G_freq(16);
G_freq(5)=1/G_freq(15);
G_freq(6)=1/G_freq(14);
G_freq(7)=1/G_freq(13);
G_freq(8)=1/G_freq(12);
G_freq(9)=1/G_freq(11);

%% Attenuation at normalized frequency f/fm
% There are three classes(0,1,2) where zero is the most demanding and 2 
% is the least demanding. Attenuations are inserted manually from the
% standard

if class == 0
alpha(1,1) = 75; alpha(1,2) = inf;
alpha(2,1) = 62; alpha(2,2) = inf;
alpha(3,1) = 42.5; alpha(3,2) = inf;
alpha(4,1) = 18; alpha(4,2) = inf;
alpha(5,1) = 2.3; alpha(5,2) = 4.5;
alpha(6,1) = -0.15; alpha(6,2) = 4.5;
alpha(7,1) = -0.15; alpha(7,2) = 1.1;
alpha(8,1) = -0.15; alpha(8,2) = 0.4;
alpha(9,1) = -0.15; alpha(9,2) = 0.2;
alpha(10,1) = -0.15; alpha(10,2) = 0.15;
alpha(11,1) = -0.15; alpha(11,2) = 0.2;
alpha(12,1) = -0.15; alpha(12,2) = 0.4;
alpha(13,1) = -0.15; alpha(13,2) = 1.1;
alpha(14,1) = -0.15; alpha(14,2) = 4.5;
alpha(15,1) = 2.3; alpha(15,2) = 4.5;
alpha(16,1) = 18; alpha(16,2) = inf;
alpha(17,1) = 42.5; alpha(17,2) = inf;
alpha(18,1) = 62; alpha(18,2) = inf;
alpha(19,1) = 75; alpha(19,2) = inf;

elseif class == 1
%% Attenuation at normalized frequency f/fm
alpha(1,1) = 70; alpha(1,2) = inf;
alpha(2,1) = 61; alpha(2,2) = inf;
alpha(3,1) = 42; alpha(3,2) = inf;
alpha(4,1) = 17.5; alpha(4,2) = inf;
alpha(5,1) = 2.0; alpha(5,2) = 5;
alpha(6,1) = -0.3; alpha(6,2) = 5;
alpha(7,1) = -0.3; alpha(7,2) = 1.3;
alpha(8,1) = -0.3; alpha(8,2) = 0.6;
alpha(9,1) = -0.3; alpha(9,2) = 0.4;
alpha(10,1) = -0.3; alpha(10,2) = 0.3;
alpha(11,1) = -0.3; alpha(11,2) = 0.4;
alpha(12,1) = -0.3; alpha(12,2) = 0.6;
alpha(13,1) = -0.3; alpha(13,2) = 1.3;
alpha(14,1) = -0.3; alpha(14,2) = 5;
alpha(15,1) = 2; alpha(15,2) = 5;
alpha(16,1) = 17.5; alpha(16,2) = inf;
alpha(17,1) = 42; alpha(17,2) = inf;
alpha(18,1) = 61; alpha(18,2) = inf;
alpha(19,1) = 70; alpha(19,2) = inf;
    
elseif class == 2
%% Attenuation at normalized frequency f/fm
alpha(1,1) = 60; alpha(1,2) = inf;
alpha(2,1) = 55; alpha(2,2) = inf;
alpha(3,1) = 41; alpha(3,2) = inf;
alpha(4,1) = 16.5; alpha(4,2) = inf;
alpha(5,1) = 1.6; alpha(5,2) = 5.5;
alpha(6,1) = -0.5; alpha(6,2) = 5.5;
alpha(7,1) = -0.5; alpha(7,2) = 1.6;
alpha(8,1) = -0.5; alpha(8,2) = 0.8;
alpha(9,1) = -0.5; alpha(9,2) = 0.6;
alpha(10,1) = -0.5; alpha(10,2) = 0.5;
alpha(11,1) = -0.5; alpha(11,2) = 0.6;
alpha(12,1) = -0.5; alpha(12,2) = 0.8;
alpha(13,1) = -0.5; alpha(13,2) = 1.6;
alpha(14,1) = -0.5; alpha(14,2) = 5.5;
alpha(15,1) = 1.6; alpha(15,2) = 5.5;
alpha(16,1) = 16.5; alpha(16,2) = inf;
alpha(17,1) = 41; alpha(17,2) = inf;
alpha(18,1) = 55; alpha(18,2) = inf;
alpha(19,1) = 60; alpha(19,2) = inf;
end
\end{lstlisting}

\begin{lstlisting}[language=Matlab, caption = {Matlab script for FIR coefficients},label={ls:matlabFirFunction}]
function [bQ, M, beta] = FIRFunction(f_pass,a_pass,f_stop,a_stop,Fs)
%% Script for calculating quantizised b coefficents 
% The kaiser window method is taken from oppenheim p. 569
%% Conversions:
omega_pass  = f_pass*2*pi/(Fs);     % The frequency F1 in radians
delta_1     = 10^(a_pass/20);       % The gain at the frequency F1
omega_stop  = f_stop*2*pi/(Fs);     % The frequency F2 in radians
delta_2     = 10^(a_stop/20);       % The gain at the frequency F2

%% Determine cutoff
omega_c = (omega_pass+omega_stop)/2; % Calculate the normalized cutoff 
fc = omega_c*Fs/2;                   % Calculate the cutoff frequency

%% Detemine kaiser window
%Find the beta value
delta_omega = omega_stop-omega_pass     % Calculate the difference
A = -20*log10(min([delta_1 delta_2]))   % Calculate the largest attenuation 

% Calculate the beta value
if A > 50 
    beta = 0.1102*(A-8.7);
elseif 21 <= A <= 50 
    beta = 0.5842*((A-21)^0.4)+0.07886*(A-21);
elseif A < 21
    beta = 0;
end

% If the beta value is below 4 set it to 4.
% This is done because the filter will create noise if beta is under 4
if beta < 4
    beta = 4;
end

%Find order
M = (A-8)/(2.285*delta_omega) % Calculate the precise order
M = 2.*round(M/2);            % Round the order to nearest eqaul value to
                              % get type 1 FIR filter

%% Impulse response
for n=0:M

% The ideal response is computed according to:
%            sin(omega_c(n-M/2)) 
%  h_d[n] = -------------------
%              omega_c(n-M/2)
 
h(n+1) = (sin(omega_c*(n-(M/2))))/(pi*(n-(M/2))); % Calculate impulse samples
end
%L hopspital regel for sin x / x
h((M+2)/2) = omega_c*cos(omega_c*((M/2)-(1/2)*M))/pi;

%Window is applied
w = kaiser(M+1,beta)'; % Calculate the kaiser window 
b = h.*w;              % Apply window


%% Quantizing
% Function for quantizising coefficients
[bq aq] = QuantizingFunction(b,1); 
bQ = bq; % Outpu
\end{lstlisting}

\begin{lstlisting}[language=Matlab, caption = {Matlab script for quantisizing FIR coefficients},label={ls:quan}]
function [numOut denOut] = QuantizingFunction(numIn,denIn)
num = numIn.*2^15;
den = denIn.*2^15;

num = round(num);
den = round(den);

numOut = num/2^15;
denOut = den/2^15;
end
\end{lstlisting}