\chapter{IEC 6964 (2001): ELECTROACOUSTICS - OCTAVE-BAND AND FRACTIONAL-OCTAVE-BAND FILTERS} \label{app:IEC6964}
This appendix will explain the main points of the standard IEC 6964 (2001) which are used to specify the requirements for the filters used in the octave-band and fractional-octave-band filters.

The standard provides performance requirements for bandpass filters which comprises a spectrum analyser. These requirements are relative attenuation to a center frequencies for a given filter set and bandwidth. This gives means that a spectrum analyzer may contain any number of bandpass filters covering a desired frequency ranged as long as the requirements are upheld. 

There are three filter classes named class 0, class 1 and class 2 which sets the tolerances of the filter requirements where a higher class equals higher tolerances.  

\textbf{Definitions} \\
G: Octave ratio \\
b: Bandwidth designator (How many bands per octave) \\
$f_r$: Highest filter center frequency \\
$f_m$: Exact center frequency of all bandpass filters

To calculate the exact mid-band frequencies of the bandpass filters the following equation is used
\begin{equation}
f_m=(G^{x/b})(f_r) \enhed{Hz}
\end{equation}
\begin{where}
\va{$x$}{is the filter number going from 0 to number of filters}{.}
\end{where}

when b is an even number the following equation should be used
\begin{equation}
f_m=(G^{(2x+1)/(2b)})(f_r) \enhed{Hz}
\end{equation}

To calculate some higher normalized frequencies to the center frequency,  where the attenuations also will be calculated, the following equation is used, for $\Omega > 1$:
\begin{equation}
\Omega_{high}(k) = 1+\Big[(G^{1/(2b)}-1)/(G^{1/2}-1)\Big](\Omega-1)
\end{equation}
\begin{where}
\va{$\Omega$}{equals $f/f_m$}{.}
\end{where}

To calculate the lower normalized frequencies of the band the reciprocal values are found by:
\begin{equation}
\Omega_{low} = \frac{1}{\Omega_{high}}
\end{equation}

For calculating the attenuations of these calculated frequencies the following equation is used:
\begin{equation}
\triangle A_x=\triangle A_a+[\triangle A_b-\triangle A_a][log(\Omega_x/\Omega_a)/log(\Omega_b/\Omega_a)]
\end{equation}
\begin{where}
\va{$\triangle A_a$}{is a relative attenuation limit at normalized frequency $\Omega_a$}{dB}
\va{$\triangle A_b$}{is a relative attenuation limit at normalized frequency $\Omega_b$}{dB}
\va{$\Omega_a$}{equals $\Omega_{low}$}{.}
\va{$\Omega_b$}{equals $\Omega_{high}$}{.}
\end{where}

A table can then be set up with relative frequencies and attenuations. The calculations of these numbers are done using the function StandardFunction2 in matlab. A normalized frequency table for the requirements is as seen on \autoref{tb:IEC6964_att_freq}. 

% Please add the following required packages to your document preamble:
% \usepackage{multirow}
\begin{table}[]
\centering
\begin{tabular}{|c|c|c|c|}
\hline
\multirow{3}{*}{\begin{tabular}[c]{@{}c@{}}Normalized\\ frequency\\ $f/f_m=\Omega$\end{tabular}}       & \multicolumn{3}{c|}{\begin{tabular}[c]{@{}c@{}}Minimum; Maximum attenuation limits\\ dB\end{tabular}}                   \\ \cline{2-4} 
                                                                                                       & \multicolumn{3}{c|}{Filter class}                                                                                       \\ \cline{2-4} 
                                                                                                       & 0                                      & 1                                      & 2                                     \\ \hline
$G^0$                                                                                                  & -0,15; +0,15                           & -0,3; +0,3                             & -0,5;+0,5                             \\ \hline
$G^{\pm 1/8}$                                                                                          & -0,15; +0,2                            & -0,3; +0,4                             & -0,5; +0,6                            \\ \hline
$G^{\pm 1/4}$                                                                                          & -0,15; +0,4                            & -0,3; +0,6                             & -0,5; +0,8                            \\ \hline
$G^{\pm 3/8}$                                                                                          & -0,15; +1,1                            & -0,3; +1,3                             & -0,5; +1,6                            \\ \hline
$<G^{\pm 1/2}$                                                                                         & -0,15; +4,5*                           & -0,3; +5,0*                            & -0,5; +5,5*                           \\ \hline
$>G^{\pm 1/2}$                                                                                         & +2,3; +4,5*                            & +2,0; +5,0*                            & +1,6; +5,5*                           \\ \hline
$G^{\pm 1}$                                                                                            & +18,0; +$\infty$                       & +17,5; +$\infty$                       & +16,5; +$\infty$                      \\ \hline
$G^{\pm 2}$                                                                                            & +42,5; +$\infty$                       & +42; +$\infty$                         & +41; +$\infty$                        \\ \hline
$G^{\pm 3}$                                                                                            & +62;+$\infty$                          & +61; +$\infty$                         & +55; +$\infty$                        \\ \hline
$\geq G^{\pm +4}$                                                                                      & +75; +$\infty$                         & +70; +$\infty$                         & +60; +$\infty$                        \\ \hline
$\leq G^{\pm -4}$                                                                                      & +75; +$\infty$                         & +70; +$\infty$                         & +60; +$\infty$                        \\ \hline
\multicolumn{4}{|l|}{\begin{tabular}[c]{@{}l@{}}* At frequencies less than the lower band-edge frequency and \\ greater than the upper band-edge frequency, the limit on \\ maximum relative attenuation is +$\infty$;\end{tabular}} \\ \hline
\end{tabular}
\caption{standard for IEC6964 base 2 with different classes}
\label{tb:IEC6964_att_freq}
\end{table}





