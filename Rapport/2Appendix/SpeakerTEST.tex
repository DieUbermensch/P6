\chapter{Speaker vibration test} \label{app:journal_speaker_test}

\section{Purpose}

This experiment is conducted in order to establish proof of a correlation between the vibration in a speaker an a increase in distortion. The Distortion level is to be viewable by spectral estimation of accelerometers placed inside the speaker enclosure. It is desirable to find the situation where the woofer hits its coil, since the is the least wanted situation of a playing speaker. The distortion or vibration measured from the speaker can later be used to regulate the input and prevent further distortion to appear. The purpose is expressly formulated as:
\begin{itemize}
\item Measure the correlation between distortion and the vibration in the enclosure
\item Find the situation where the woofer hits the coil. That is at;
\begin{itemize}
\item What frequency 
\item What Amplitude
\end{itemize}
\item [] It occurs 
\end{itemize}

\section{Set-up}

The set-up of this experiment are depicted on Figure \ref{figure:SpeakertestSetup} and described as follows:
\begin{itemize}
\item Distortion will be measured by a microphone at a distance 1 meter
\item Vibration will be measured by a Brüel \& Kjear Type \textbf{TBD}, placed at:
\begin{itemize}
\item Both Woofers
\item Centrally inside the enclosure 
\item The PCB of the crossover
\end{itemize}
\item \textit{A microphone will be placed centrally inside the speaker to measure distortion}
\item The speaker will be driven by a \textbf{TBD} Amplifier.
\item A computer will be logging data with a \textbf{TBD} soundcard
\end{itemize}

Furthermore the speaker will be placed in the anechoic room to eliminate any external disturbances. 


\begin{figure}[H]
\centering
\missingfigure{Picture of setup}
\caption{test setup}
\label{figure:SpeakertestSetup}
\end{figure}

\subsection*{Equipment used and AAU-no.}

\begin{table}[H]
\centering
\ra{1.3}
\begin{tabular}{S[table-format=1]ccc} \toprule
    {Item} & {Description} & {AAU-no} \\ \bottomrule 
    1      &  B\& K Accelerometer Type 4344  & 64659   \\ 
    2      &  B\& K Accelerometer Type 4344  & 64660   \\ 
    3      & YY  & XX   \\
    4      & YY  & XX   \\ 
    5      & YY  & XX  \\ \bottomrule 
\end{tabular}
\caption{Table over used equipment}
\end{table}



\section{Procedure}\label{sec:SpeakerTestProcedure}

The producer for this experiment is described as follows:
\begin{enumerate}
\item Adjust volume on pre-amplifier.
\item Capture a sinusoidal sweep from 2.4 Khz to 10 Hz.
\item If the speaker could handle it, increase the volume with 3 dB.
\item Repeat on the woofer hits coil. 
\end{enumerate}

All of the accelerometers have beforehand been calibrated to have the same sensitivity within the frequency test range.


\section{Data Extraction}

The information captured by the use of procedure in \autoref{sec:SpeakerTestProcedure} is graphed in this section


\section{Analysis}

Comments and potential further manipulation is documented in this section. 

\section{Error sources}

Potential Error sources are noted in the section

