\chapter{DSP setup}

In this chapter the setup of the DSP is explained.


\section{DSP setup}


\begin{lstlisting}[language=C, caption = {System initialization},label={listingSystemInit}]
void system_init(uint8 audioType, uint8 resolution, uint8 fs){
    SYS_PCGCR1 	 = 0x0000;     		/* Enable clocks to all peripherals */
    SYS_PCGCR2 	 = 0x0000;
	SYS_EXBUSSEL = 0x1100;         	// Enable I2S bus
	I2C_init();        				// Initialize I2C
	
	if 		(fs == 48 && resolution == 16) TLV320AIC3204_init(0x0d, 0x07, 0x06, 0x90);
	else if (fs == 48 && resolution == 24) TLV320AIC3204_init(0x2d, 0x07, 0x06, 0x90);	
	else if (fs == 96 && resolution == 16) TLV320AIC3204_init(0x0d, 0x0E, 0x0D, 0x20);	
	else if	(fs == 96 && resolution == 24) TLV320AIC3204_init(0x2d, 0x0E, 0x0D, 0x20);
	else TLV320AIC3204_init(0x0d, 0x07, 0x06, 0x90);	
	
	wait(200);        				// Wait	
	if 		(audioType == 0 && resolution == 16) I2S_init(0x9010);
	else if (audioType == 0 && resolution == 24) I2S_init(0x901C);	
	else if (audioType == 1 && resolution == 16) I2S_init(0x8010);	
	else if	(audioType == 1 && resolution == 24) I2S_init(0x801C);
	else I2S_init(0x9010);
	
}

\end{lstlisting}

\begin{lstlisting}[language=C, caption = {Setup of I2S port for the DSP},label={listingI2SDSP}]
void I2S_init(uint8 Type){
	I2S0_SRATE 		= 0x0;
    I2S0_CTRL 		= Type;    	// 16-bit word, slave, enable I2S (0x8010), stereo. 24-bit word, slave, enable I2S, mono (0x901C). 
    I2S0_INTMASK 	= 0x3F;    		// Enable interrupts (Stereo 0x2B, Mono 0x17)
}
\end{lstlisting}


\section{TLV320AIC3204 setup}


\section{UART setup} \label{UART_setup}

\begin{lstlisting}[language=C, caption = {Initialization of UART},label={listingUartInit}]
// By 16gr640, Spring 2016, AAU
#include "stdio.h"
#include "uart.h"
#include "ezdsp_setup.h"
#define INPUT_FRERQUENCY 100000000

void uartInit(long baudRate)
{
	
	int16 baudRateLSB;
	int16 baudRateMSB;
	
	SYS_EXBUSSEL = 0x1100;
	
	// Setup power management (PWREMU_MGMT). Setting UTRST and URRST to 0
	UART_PWREMU_MGMT = 0x0000;
	
	// Baud rate setup 
	UART_LCR = 0x80; // Setup control line register to change baud rate
	
	// Calculate baud rate
	baudRate = INPUT_FRERQUENCY/(baudRate*16);
	baudRateLSB = (baudRate&0xFF);
	baudRateMSB = ((baudRate>>8)&0xFF);
	
	UART_DLL = baudRateLSB; // Baud rate = 9600
	UART_DLH = baudRateMSB;
	UART_LCR = 0x00; // Disable control line register to not change baud rate
	
	UART_IER = 0x01;
	
	// Setup FIFO control register. (FIFOEN set first) 
	UART_FCR = 0x07; // Clear FIFO's and activate FIFO mode
	
	// Choosing desired protocol setting in LCR
	UART_LCR = 0x03; // No parity bits and 8 bit length word
	
	// (free bit setting in power management and activate UART)
	UART_PWREMU_MGMT = 0x7FFF; // Go go go
	
}

void uartWrite(char *string)
{
	int16 cnt;
	for(cnt=0;string[cnt]!=NULL;cnt++)
	{
		serialWriteByte = (int16)string[cnt];
	}
}
\end{lstlisting}

\begin{lstlisting}[language=C, caption = {Set a flag high if data available in FIFO},label={listingUartFlag}]
    	if(UART_AVAILABLE == 1)
    	{	
			uartFlag = 1;
		}
\end{lstlisting}

\begin{lstlisting}[language=C, caption = {Read the incomming packet from FIFO buffer},label={listingUartRead}]
		/////// UART READ //////
		if (uartFlag == 1){	
			uartReceiveData[0] = UART_RBR;
			uartReceiveData[1] = UART_RBR;
			uartReceiveData[2] = UART_RBR;
			if ((long)uartReceiveData[0] == (long)uartReceiveData[2])
			{
				switch(uartReceiveData[0])
				{
					case 1:
						gainDown128 = uartReceiveData[1];
						break;
					case 2:
						gainBand128 = uartReceiveData[1];
						break;
					case 3:
						gainBand64 = uartReceiveData[1];
						break;
					case 4:
						gainBand32 = uartReceiveData[1];
						break;
					case 5:
						gainBand16 = uartReceiveData[1];
						break;
					case 6:
						gainBand8 = uartReceiveData[1];
						break;
					case 7:
						gainBand4 = uartReceiveData[1];
						break;
					case 8:
						gainBand2 = uartReceiveData[1];
						break;
					case 9:
						gainBand1 = uartReceiveData[1];
						break;
					case 10:
						volume = uartReceiveData[1];
						break;
					case 11:
						byPass = uartReceiveData[1];
						break;
					default: break;
				}	
			} 
			else 
			{
				uartReceiveData[0] = 0;
				uartReceiveData[1] = 0;
				uartReceiveData[2] = 0;
			}
			uartFlag = 0;
		}	
\end{lstlisting}

\begin{lstlisting}[language=C, caption = {Transmit the RMS values of all bands. Only runs every second time},label={listingUartTransmit}]
		///// UART TRANSMIT RMS //////
		if (uartTransmitFlag == 1){
			serialWriteByte = 1;
			serialWriteByte = RMS64bandValue;
			serialWriteByte = RMS32bandValue;
			serialWriteByte = RMS16bandValue;
			serialWriteByte = RMS8bandValue;
			serialWriteByte = RMS4bandValue;
			serialWriteByte = RMS2bandValue;
			serialWriteByte = RMS1bandValue;
			serialWriteByte = 1;
			uartTransmitFlag = 0;
		} else {
			uartTransmitFlag = 1;
		}
\end{lstlisting}
