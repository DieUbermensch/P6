\section{List of Requirements \label{requirements}}
Based upon the previous considerations, requirements for the segway can be determined. The requirements are as follows:

\subsection*{Functional Requirements}
\begin{enumerate}
\item The segway must be able to stabilise itself in an upright equilibrium state.\label{funcReqBalance}
\item The segway must be able to drive forward and backwards without causing the segway to fall over.
\item The segway must be able to turn without causing the segway to fall over.
\item The segway may not fall over when pushed lightly.
\item The user must be able to make the segway drive by a wireless controller.
\item The user must be able to make the segway turn by a wireless controller.
\item The user must be able to receive data from the segway on a wireless controller.
\end{enumerate}
\subsection*{Performance Requirements}
The following requirements describes the desired performance of the segway. The first list of requirements is for the time domain, determining the dynamic behaviour of the segway. %Note that these requirements are made based on choices made by the project group and therefore chosen.
\begin{itemize}
\item Steady state error: 0 \%
\item Rise time: $\leq$ 1 s
\item Settling time: $\leq$ 3 s
\item Overshoot: $\leq$ 10 \% 
\end{itemize}
The following list describes the desired stability margins, i.e. the frequency domain requirements:
\begin{itemize}
\item Phase margin: $\geq 80\degree$
\item Gain margin: $\geq$ 6 dB
\end{itemize}

Now that the requirements are set, the development of the system can begin. This includes designing the controller which is to stabilize the system, together with a filter design and the design of the remote controller and the communication protocol. In the following section, a model of the system is derived, as this is needed to make a controller for the system.
%\begin{itemize}
%\item Rise time: TBD
%\item Settling time: TBD
%\item Overshoot: TBD
%\item Steady state error: TBD
%\item 
%
%\end{itemize}