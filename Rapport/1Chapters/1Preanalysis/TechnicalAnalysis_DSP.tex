\section{Processor Platform}
This section will reference the demands for the processor platform given by DALI and analyse which processor platform will be optimal for these condition. 

\subsection*{Demands for the platform}
The demands for the processor platform have been derived in a mail corresodance with DALI 
\todo[inline]{Reference til CD}

\textbf{Sample rate} \\
The system must be able to handle 96 kHz sample rate since peripheral components will interface be running at this rate.

\textbf{Bit resolution} \\
There must be support for audio in at least a 24 bit resolution since the \gls{DAC} chosen by DALI is of 24 bit.

\textbf{\gls{SNR}} \\
A minimum \gls{SNR} of 120 dB on the \gls{DAC} must be achieved. This will provide headroom for digital volume control.

\textbf{Instructions pr. sample.} \\
The processor must a least be able to run 1024 intructions pr. sample. which would give approximately 100 \gls{MIPS}


\textbf{Memory for both an equalizer and multi band limiter} \\
The platform will consist of an equalizer and limiter when finished. This project will however focus on the equalizer and sensory system. Hence it only desireable to leave TBD room for an multi band limiter which is desired by DALI. 


\textbf{Interfacing} \\
The platform must be able to interface with an ADC and DAC with the bus type \gls{I2S}. Besides the interfacing with the ADC and DAC the platform must be able to interface with the chosen sensor in \autoref{sec:Sensor}.

The 32 bit resolution is prefered by DALI over a 24 bit resolution because it would give a increased \gls{SNR} of 48 dB. These demands lead to the analyzation of which platform would be the most optimal.   

\subsection*{Choice of processor platform}
For the choice of processor platform different types of platforms could be used, such as a microcontroller or a \gls{DSP}. These two types of processors have different advantage and disadvantages which needs to be analyzed to find the most optimal platform for the demands above.

The basic difference between a microcontroller and a \gls{DSP} is that the \gls{DSP} is designed for fast calculating and moving data while the microcontroller is designed to be more flexible in its use and have more features.

\begin{table}[H]
\centering
\begin{tabular}{lllll}
\toprule
 & \multicolumn{1}{c}{ASIC} & \multicolumn{1}{c}{FPGA} & \multicolumn{1}{c}{$\mu$P/$\mu$C} & \multicolumn{1}{c}{\begin{tabular}[c]{@{}c@{}}Digital signal\\ processor\end{tabular}} \\ \hline
\textbf{Flexibility} & None & Limited & High & High \\
\textbf{Design time} & Medium & Medium & Short & Short \\
\textbf{Power consumption} & Low-medium & Medium-high & Medium high & Low-medium \\
\textbf{Performance} & High & Low-medium & Low-medium & Medium-high \\
\textbf{Development cost} & Medium & Low & Low & Low \\ 
\textbf{Production cost} & Low-medium & Medium-high & Medium-high & Low-medium \\ \bottomrule 
\end{tabular}
\caption{Summary of DSP hardware implementations \citep{WileyDSP}.}
\label{tb:summary_DSP_hardware_implementation}
\end{table}

Because the need for filtering in an equalizer is needed and the high performance demands from above the natural choice of platform would be a \gls{DSP} without further consideration.     




