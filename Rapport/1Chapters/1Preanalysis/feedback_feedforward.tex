\section{Real-time Signal Processing System}

It is important to clarify if the processing system is going to be designed as an analog or digital system. Both types have its own advantages and disadvantages and one of the purposes this section is thus to determine which platform will suit the system most. A brief analysis on the advantages and disadvantages of both system are therefore provided. 

Briefly, the main difference between an analog and digital system is that the analog system works in continuous-time while the digital system works in discrete-time. For a continuous-time system values of a continuous signal, at any given time, is defined. For a discrete-time system, values of a signal are only defined at specific points in time with a fixed resolution. The time difference between each points is defined through defined sampling period $T_s$ and the resolution of each point is defined by the resolution of the Analog-to-Digital Converter (ADC). By using a digital system, some information of the signal is necessarily lost, since there will be points in time where the signal is not defined.

For an analog system, components such as resistors, capacitors, inductors, transistors and so on, are usually used. A digital system will typically consist of an ADC to discretize the analog signal, a processor to process the signal, and a Digital-to-Analog Converter (DAC) to convert the signal back to an analog signal. The advantage of designing an analog system is that ideally no information is lost. This conclusion is however derived with the assumption that analog components are ideal since distortion will arise as soon as non-ideal components are used. The problem with ideal components is not a problem for a digital system, since no further distortion will be applied to the signal as soon as the signal is discretized.

As the system to be developed, most likely will rely on signal processing, the great disadvantages of an analog system is the constraints and flexibility to apply signal processing compared to a digital system. Some applications such as the Fourier Transformation is very hard if not impossible to realize in an analog system while realizable in a digital system through algorithms. Also, if the system is large, an analog system might turn out to be very complex, and use more components which results in more distortion. As the system to be developed is relying on signal processing the platform is chosen to be digital.

\section{Feedback versus Feedforward System}

At this point it is yet to be determined if the system will be based on a feedback or a feedforward system. The choices are to design a feedback system that will decide what to do, based on real-time measurements from a sensor i.e. accelerometer, or a feedforward system that will analyse an input signal and then decide what to do. Previous tests have shown that it is possible to use an accelerometer to estimate the performance of the system by looking at the harmonic distortions. A big issue is to analyse harmonic distortions when music is played back, as estimating harmonic distortions introduced in a system, are only well-defined for periodic sinusoids. The tests of the loudspeaker also showed that the frequency response of vibration on the driver and cabinet hardly changes when exposed for loud playback, thus it will be hard to derived anything alone from the frequency response of the vibrations. It can therefore be concluded that developing a system based on measurements from a accelerometer might turn out to be too complex to be realized in time due to time constraints. The system to be designed will therefore be based on a feedforward system. The concept of a feedback system which fulfil the problem statement is shown in \autoref{fig:Concept}.

\begin{figure}[H]
\centering
\tikzsetnextfilename{Concept}
\scalebox{0.8}{
\begin{tikzpicture}
%% Kasser %%
\node [block, fill=black!80,text=white] (AudioSource) at (0,0) {Audio Source};

%% DSP %%
\node [block, fill=blue!15] (Equalizer1) at ($(AudioSource)+(3.5,0)$) {Equalizer};
\node [block, fill=blue!15] (SpectralAnalysis) at ($(Equalizer1)+(3.5,0)$) {Signal analysis};
\node [block, fill=blue!15] (Equalizer) at ($(SpectralAnalysis)+(4.5,0)$) {Signal processing};


%% External %%
\node [block, fill=blue!15] (UserInterface) at ($(Equalizer1)+(-3.5,2)$) {User interface};
\node [block, fill=blue!15] (UserInterfaceDev) at ($(Equalizer1)+(3.5,2)$) {Developer interface};

%% Speaker Enclosure %%
\node [block, fill=black!80,text=white] (Amplifier) at ($(SpectralAnalysis)+(4.5,-3)$) {Amplifier};
\node [block, fill=black!80,text=white] (Driver) at ($(Amplifier)+(-3.5,0)$) {Driver};

\node (User) at ($(-2.5,0)+(UserInterface)$) {\textbf{User}};
\node (Developer) at ($(3.6,0)+(UserInterfaceDev)$) {\textbf{Developer}};
\draw[->] (User) -- (UserInterface);
\draw[->] (Developer) -- (UserInterfaceDev);

% Store Blokke %%
\begin{pgfonlayer}{bg}
\draw[thick, fill=black!10] ($(-1.75,1.2)+(Equalizer1)$) -- ($(2.5,1.2)+(Equalizer)$) -- ($(2.5,-1.5)+(Equalizer)$) -- ($(-1.75,-1.5)+(Equalizer1)$) -- ($(-1.75,1.2)+(Equalizer1)$);
\node (DSPtag) at ($(0,-1.1)+(SpectralAnalysis)$) {\textbf{Digital Processing System}};

\draw[thick, fill=black!10] ($(-2.5,1)+(Driver)$) -- ($(2.5,1)+(Amplifier)$) -- ($(2.5,-1.5)+(Amplifier)$) -- ($(-2.5,-1.5)+(Driver)$) -- ($(-2.5,1)+(Driver)$);
\node (DSPtag) at ($(2,-1.1)+(Driver)$) {\textbf{Speaker Enclosure}};
\end{pgfonlayer}

%% Forbindelse %%
\draw[->] (AudioSource) -- (Equalizer1);
\draw[->] (Equalizer1) -- (SpectralAnalysis);
\draw[->] (SpectralAnalysis) -- (Equalizer);
\draw[->] (Equalizer) -- (Amplifier);
\draw[->] (Amplifier) -- (Driver);

\draw[->] (UserInterface) -| ($(Equalizer1)+(-.25,.6)$);
\draw[->] (UserInterfaceDev) -| ($(Equalizer1)+(.25,.6)$);

%\draw[->] ($(SensorDriver)+(-1.375,-0.3)$) -- ($(SpectralAnalysis)+(1.375,-0.3)$);

%\draw[->] (SpectralAnalysis) -- (Decisionblock);
%\draw[->] (Decisionblock) |- (Equalizer);
\end{tikzpicture}}
\caption{Overview of design concept.}
\label{fig:Concept}
\end{figure}

The design concept consist of a signal processing block, which will be analysed to understand which solution is best for the problem stated, a signal analysis block which will not be analysed further in this chapter, a user interface which will be analysed to determine how the user interface should be designed and lastly the processor block which will be analysed to determine which type of processor that should be used. 









%As the audio signal must be continuous when played on loudspeakers