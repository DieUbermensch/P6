\section{Signal Analysis}

After the signal has been processed in an equalizer, the signal will be analysed in the \textit{Signal Analysis}-block. The purpose of the signal analysis is to examine the content of the signal and afterwards send the parameters to the signal processing subsystem. Generally speaking, any signal can be analysed in time domain or frequency domain. If analysed in time domain, the analysis can for instance be extracting signal feature and calculating of the signal's RMS value. Analysing the signal in frequency domain will reveal the spectral content of the signal. So the following section will cover how different techniques for extracting information from the signal in both time and frequency domain.

\begin{itemize}
\item[•] Signal feature extraction.
\item[•] Spectral analysis.
\end{itemize}

\subsection{Signal feature extraction}
% Signal analysis in time domain

For a continuous signal the RMS value is defined as:
\begin{equation}
V_{\text{RMS}}(t) = \sqrt{\frac{1}{T}\int_0^T v(t)^2 dt}
\end{equation}

For a discrete signal:
\begin{equation}
V_{\text{RMS}}[n] = \sqrt{\frac{1}{N}\sum_{k=1}^{N} v[n-k]^2 dt}
\end{equation}


\subsection{Spectral analysis}
% Signal analysis in frequency domain

























