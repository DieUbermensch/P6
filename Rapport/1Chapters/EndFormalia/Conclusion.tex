\chapter{Conclusion}\label{cha:conclusion}
This project has revolved around creating a driver protection system for a loudspeaker system in cooperation with DALI. DALI provided the platform and desired requirements for the system. The goal of the system was to reduce the chance of the coil hitting the back plate.

The initial solution consisted of a feedback system utilizing accelerometer input to correct the signal according to the movement of the driver. It was possible to measure the response of the driver with precise enough instrumentation but the signals became to distorted when cheaper instrumentation was used. Further analysis of the measured data showed that analyzing continuous multi-tone signals were problematic since every signal were of a random nature. Furthermore using a specific frequency could resolve in false positives and negatives since out of phase signals could change the signal which were to be feed back. The solution was however discarded as a plausible solution due to complexity and time constraints.

It was determined to base the solution on a feedforward system. This was mainly due to the fact that the hit was present at random times and hence the feedback system was only able to regulate when the hit had taken place. Using already analyzed microphone measurements and accelerometer data it was possible to determine a model for THD according to a given RMS value. With these models and recorded scenarios of back plate hits it was possible to determine parameters for limiting the signal feed to the speaker. 

 The system was implemented with FIR filters to reduce noise and utilize the linear phase for spectral inversion. The spectral inversion along with a multi-rate system made it possible to realize a 4 band octave spectrum analyzer in the low frequency area of 500 Hz and downwards. Each band was fitted with an RMS algorithm which would measure the signal level of the specific bands. These bands are then regulated using a look up table which contained the needed attenuation for the signal not exceed a desired threshold. The threshold was determined such the speaker wont be subjected to very high currents when played at high SPL, which increases the chance of a hit. A first order IIR filter was then used to determine attack and release time such the limiter would be less audible .
 
 The system was however not implemented correctly which resulted in gain variation in the limiter. Combined with the resolution of the look up table not being scaled properly it resulted in audible attenuation changes and not correct attenuation of the signal. It was however possible to create a limiter which attenuated enough to avoid a backplate hit at 150 W.