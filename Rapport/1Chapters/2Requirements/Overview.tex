\chapter{Overview} \label{ch:overview}
This chapter will describe an overview of the system before the needed requirements of the system is determined. Recall the overview of system from \autoref{fig:Concept}. It was concluded from the previous chapter to use a spectrum analyzer combined with a limiter, but it has yet to be determined if it should be a single-band limiter or a multi-band limiter. So firstly this will be examined, while a final overview of the system will be given afterwards. 

%This block diagram is know futher extended to the block diagram seen in \autoref{fig:SystemOverview}, based on the conclusions from \autoref{sec:PreAnalysisCon}.

%\section{Single-band vs multi-band system}
%This section will examine if the system should use multi-band or single-band. When the system is to be designed a question whether one or more bands should be implemented arises. 
There are advantages and disadvantages with both a single-band and a multi-band limiter, but the question arises which sounds better?
A multi-band solution could create an unnatural frequency response, making the music sounding unpleasant. If one of the bands are attenuated to much it could result in a band acting like a notch filter which could sound strangely. The single-band solution will be attenuating every frequency keeping the amplitude relation intact. However the single-band will also be attenuating many frequencies which are not contributing to the problem. In order to determine which solution is to be preferred, a listening test is conducted with both a single-band and a multi-band limiter, to conclude what type of limiter sounds better. 

In the test a group of 10 people were subjected to the following effect blocks:
 \begin{itemize}
    \item Single-band limiter:
    \begin{itemize}
    \item 1 limiter ranging the entire frequency spectrum.
    \item 12 dB attenuation at 0 dBFS.
    \item Attack time of 0.5 ms and release time of 1 ms.
    \end{itemize}
    \item Multi-band limiter:
    \begin{itemize}
    \item 1 limiter ranging from 0 to 100 Hz.
    \item 1 limiter ranging from 100 to 300 Hz.
    \item 1 limiter from 300 to 1000 Hz.
    \item 1 limiter ranging the entire frequency spectrum.
    \item Attack time of 0.5 ms and release time of 1 ms.
    \item 12 dB attenuation at 0 dBFS.
    \end{itemize}
 \end{itemize}
Using the following songs in different genres:
\begin{itemize}
\item Shots \& Squats by Vigiland (Dance)
\item Sultans of Swing by Dire Straits (Classic rock)
\item Stormy Weather by Teitur (Folk)
\item Hearth of Courage by Two Steps From Hell (Classical)
\end{itemize}
The test concluded with the following results which can be seen in \autoref{fig:piechartsongsReport}. A rather large portion of the test person were favourable of the multi-band limiter because of the weighting of the signal which ultimately created less distortion in the higher frequency spectrum than the single-band. The test is fully described in \autoref{app:journal_ListningTest}.

\begin{figure}[H]
\centering
\resizebox{0.4\textwidth}{!}{
\tikzsetnextfilename{Piechartsongs}
\begin{tikzpicture}
[
    pie chart,
    slice type={comet}{blu},
    slice type={legno}{rosso},
    slice type={coltello}{giallo},
    slice type={sedia}{viola},
    slice type={caffe}{verde},
    pie values/.style={font={\small}},
    scale=2
]
%    \pie{2008}{73/comet,13/legno,7/sedia,7/coltello}
%    \pie[xshift=2.2cm,values of coltello/.style={pos=1.1}]%
%        {2009}{52/comet,23/legno,17/sedia,3/coltello,5/caffe}
    \pie[values of caffe/.style={pos=1.1}]%
       {}{87.5/comet,10/legno,2.5/caffe}
%    \pie{2008}{87.5/comet,2.5/legno,10/sedia}
    \legend[shift={(1.5cm,0.5cm)}]{{Multiband}/comet, {Uncertain}/caffe, {Singleband}/legno}
%    \legend[shift={(3cm,-1cm)}]{{Chair (Manzano)}/sedia, {Coffee (Trieste)}/caffe}
\end{tikzpicture}}
\caption{Overall comparison of the answers. This figure is extracted from \autoref{app:journal_ListningTest}}
\label{fig:piechartsongsReport}
\end{figure}

The multi-band weighted the signal so most of the compression happened in the lower range instead of the higher frequency spectrum, this gave a thinner sound but less distortion than the single-band limiter which limited the entire frequency spectrum. From this it can be concluded that a multi-band limiter should be used. 

Because of multi-band the spectrum analyzer must be divided into several bands, so to weight each band equally the bands should be octave bands. The standard IEC 6964 (2001), refer to \autoref{app:IEC6964} for detailed description of the standard, states the requirements for octave bands in a spectrum analyzer and will therefore be used as the requirements for the bands. Using octave bands will result in 4 bands below 500 Hz which are stated in \autoref{tb:freqBands}.

\begin{table}[H]
\centering
\begin{tabular}{|
>{\columncolor[HTML]{C0C0C0}}l |c|c|c|}
\hline
& \cellcolor[HTML]{C0C0C0}$f_1$ {[}Hz{]} & \cellcolor[HTML]{C0C0C0}$f_m$ {[}Hz{]} & \cellcolor[HTML]{C0C0C0}$f_2$ {[}Hz{]} \\ \hline
Band 5 & 530,3                               & -                                   & -                                   \\ \hline
Band 4 & 265,2                               & 375                                 & 530,3                               \\ \hline
Band 3 & 132,6                               & 187,5                               & 265,2                               \\ \hline
Band 2 & 66,29                               & 93,75                               & 132,6                               \\ \hline
Band 1 & 0                                   & 46,87                               & 66,29                               \\ \hline
\end{tabular}
\caption{Frequencies for bands. ($f_1$ lower cutoff frequency of the band)($f_m$ center frequency of the band)($f_2$ higher cutoff frequency of the band)}
\label{tb:freqBands}
\end{table}

Because of the multi-bands it is also possible to have a five band graphic equalizer implemented very easily. With single-band versus multi-band concluded the following system seen on \autoref{fig:SystemOverview} is derived.
\begin{figure}[H]
\centering
\tikzsetnextfilename{SystemOverview}
\scalebox{0.8}{
\begin{circuitikz}
%% TOP BAND %% 
\node[] (AudioIN) at (-0.5,0) {Audio in};
\node[] (AudioIN) at (0,0) {};
\node[summation] (Summation) at (12.5,0) {};
\node[summation] (Summation2) at (12.5,-2) {};
\node[summation] (Summation3) at (12.5,-4) {};
\node[summation] (Summation4) at (12.5,-6) {};
\node[] (AudioOUT) at (14.5,0) {Audio out};
\draw[-] ($(AudioIN)+(1,0)$) -- ($(AudioIN)+(1,-8)$);
\draw[-] ($(AudioIN)+(1,0)$) -- ($(AudioIN)+(0.5,0)$);

%% Summation %%
\draw [->] (Summation2) -- node[at end,yshift=2mm,xshift=0mm]{\scalebox{0.5}{+}} (Summation);
\draw [->] (Summation) -- (AudioOUT);
\draw [->] (Summation3) -- node[at end,yshift=2mm,xshift=0mm]{\scalebox{0.5}{+}} (Summation2);
\draw [->] (Summation4) -- node[at end,yshift=2mm,xshift=0mm]{\scalebox{0.5}{+}} (Summation3);


%%% GUI %%
\node[gain] (GUI) at ($(7,1)+(AudioIN)$) {GUI};
\draw[->,red] ($(GUI)+(0.5,0.25)$) -- ($(GUI)+(1,0.25)$);
\draw[<-,blue] ($(GUI)+(0.5,-0.25)$) -- ($(GUI)+(1,-0.25)$);

\draw [->,red] ($(4,.25)+(AudioIN)$) -- ($(4.5,.25)+(AudioIN)$); %% add (.5,.25) to the coordinate

%% Second Band
\draw [->,red] ($(4,-1.75)+(AudioIN)$) -- ($(4.5,-1.75)+(AudioIN)$);
\draw [<-,blue] ($(7.7,-2.75)+(AudioIN)$) -- ($(8.2,-2.75)+(AudioIN)$);
%% Third Band
\draw [->,red] ($(4,-3.75)+(AudioIN)$) -- ($(4.5,-3.75)+(AudioIN)$);
\draw [<-,blue] ($(7.7,-4.75)+(AudioIN)$) -- ($(8.2,-4.75)+(AudioIN)$);
%% Fourth Band
\draw [->,red] ($(4,-5.75)+(AudioIN)$) -- ($(4.5,-5.75)+(AudioIN)$);
\draw [<-,blue] ($(7.7,-6.75)+(AudioIN)$) -- ($(8.2,-6.75)+(AudioIN)$);
%% Fifth Band
\draw [->,red] ($(4,-7.75)+(AudioIN)$) -- ($(4.5,-7.75)+(AudioIN)$);
\draw [<-,blue] ($(7.7,-8.75)+(AudioIN)$) -- ($(8.2,-8.75)+(AudioIN)$);


%% Box EQ %%
\draw [-,dashed,thick] ($(1.25,1.25)+(AudioIN)$) -| ($(6,-9.75)+(AudioIN)$) -| ($(1.25,1.25)+(AudioIN)$);
\node [] at ($(3.75,.9)+(AudioIN)$) {\textbf{Equalizer}};

\draw [-,dashed,thick] ($(6.5,-0.75)+(AudioIN)$) -| ($(11.5,-9.75)+(AudioIN)$) -| ($(6.5,-0.75)+(AudioIN)$);
\node [] at ($(9,-1)+(AudioIN)$) {\textbf{RMS Limiter}};


%% Box Compressor %%





%% First Band %% 
\draw ($(1,0)+(AudioIN)$) to[highpass] node[yshift=-5mm,xshift=-10mm, below,midway]{High Pass} ($(4,0)+(AudioIN)$);
\draw ($(4,0)+(AudioIN)$) to node[yshift=0mm,gain,label=below:Gain 5]{$G_{5}$} ($(6,0)+(AudioIN)$);
\draw [->]($(6,0)+(AudioIN)$) -- node[at end,yshift=0mm,xshift=2mm]{\scalebox{0.5}{+}} (Summation);

%% Second Band %%
\draw ($(1,-2)+(AudioIN)$) to[bandpass] node[yshift=-5mm,xshift=-10mm, below,midway]{Band 4} ($(4,-2)+(AudioIN)$);
\draw ($(4,-2)+(AudioIN)$) to node[yshift=0mm,gain,label=below:Gain 4]{$G_4$} ($(6,-2)+(AudioIN)$);
%%% Second band - Compressor%%%
\draw [->]($(6,-2)+(AudioIN)$) to node[yshift=0mm,gain,near end]{$G_{4}$} ($(-.5,0)+(Summation2)$) -- node[at end,yshift=0mm,xshift=2mm]{\scalebox{0.5}{+}} (Summation2);
\draw ($(6,-2)+(AudioIN)$) -| ($(7,-3)+(AudioIN)$) to node[yshift=0mm,gain]{RMS} ($(10.5,-3)+(AudioIN)$) -| ($(10.5,-2.52)+(AudioIN)$);

%% Third Band %%
\draw ($(1,-4)+(AudioIN)$) to[bandpass] node[yshift=-5mm,xshift=-10mm, below,midway]{Band 3} ($(4,-4)+(AudioIN)$);
\draw ($(4,-4)+(AudioIN)$) to node[yshift=0mm,gain,label=below:Gain 3]{$G_3$} ($(6,-4)+(AudioIN)$);
%%% Third band - Compressor%%%
\draw [->]($(6,-4)+(AudioIN)$) to node[yshift=0mm,gain,near end]{$G_{3}$} ($(-.5,0)+(Summation3)$) -- node[at end,yshift=0mm,xshift=2mm]{\scalebox{0.5}{+}} (Summation3);
\draw ($(6,-4)+(AudioIN)$) -| ($(7,-5)+(AudioIN)$) to node[yshift=0mm,gain]{RMS} ($(10.5,-5)+(AudioIN)$) -| ($(10.5,-4.52)+(AudioIN)$);

%% Fourth Band %%
\draw ($(1,-6)+(AudioIN)$) to[bandpass] node[yshift=-5mm,xshift=-10mm, below,midway]{Band 2} ($(4,-6)+(AudioIN)$);
\draw ($(4,-6)+(AudioIN)$) to node[yshift=0mm,gain,label=below:Gain 2]{$G_2$} ($(6,-6)+(AudioIN)$);
%%% Fourth band - Compressor%%%
\draw [->]($(6,-6)+(AudioIN)$) to node[yshift=0mm,gain,near end]{$G_{2}$} ($(-.5,0)+(Summation4)$) -- node[at end,yshift=0mm,xshift=2mm]{\scalebox{0.5}{+}} (Summation4);
\draw ($(6,-6)+(AudioIN)$) -| ($(7,-7)+(AudioIN)$) to node[yshift=0mm,gain]{RMS} ($(10.5,-7)+(AudioIN)$) -| ($(10.5,-6.52)+(AudioIN)$);
%
%% Fifth band %% 
\draw ($(1,-8)+(AudioIN)$) to[lowpass] node[yshift=-5mm,xshift=-10mm, below,midway]{Low Pass} ($(4,-8)+(AudioIN)$);
\draw ($(4,-8)+(AudioIN)$) to node[yshift=0mm,gain,label=below:Gain 1]{$G_1$} ($(6,-8)+(AudioIN)$);
%%% Fourth band - Compressor%%%
\draw [->]($(6,-8)+(AudioIN)$) to node[yshift=0mm,gain,near end]{$G_{1}$} ($(-.5,-2)+(Summation4)$) --($(0,-2)+(Summation4)$) -- node[at end,yshift=2mm,xshift=0mm]{\scalebox{0.5}{+}} (Summation4);
\draw ($(6,-8)+(AudioIN)$) -| ($(7,-9)+(AudioIN)$) to node[yshift=0mm,gain]{RMS} ($(10.5,-9)+(AudioIN)$) -| ($(10.5,-8.52)+(AudioIN)$);
\end{circuitikz}}
\caption{Block overview of the entire system.}
\label{fig:SystemOverview}
\end{figure}
%This shows a feedforward system which will equalize of the audio based on the \textbf{TBD} model, seen on \textbf{figure TBD}, comparing the second and third harmonic relative to keynote. Besides the feed forward system a user controlled equalizer should also be part of the system to give the user more flexibility. The following chapter describes the individual blocks in more detail.          
\autoref{fig:SystemOverview} shows a feedforward system with a user controlled five band graphic equalizer, controlled with a GUI, with the band frequencies seen in \autoref{tb:freqBands}. Besides the user controlled equalizer a four band RMS limiter (band 1-4) will monitor the RMS levels of the four bands and gain relative to a threshold of 150 W set in \autoref{sec:hit_detect}. The divided signal is then summed up to a single signal and outputted. This should result in a protective system which makes sure that the driver never breaks because of the coil hitting the backplate. A more detailed description of every block will now follow in the preceding section.

\textbf{Audio source} \\
The audio source in this system will be based on the IEC 60268-15 specification. Meaning that it will be limited to a analog signal with a maximum peak voltage of 2 V RMS but otherwise nominal 0.5 V RMS for a line signal. The concept system will be developed upon a mono system as a prove of concept. If a stereo system is desired the system can be duplicated. %This input will handled by an onboard TLV320AIC3204 codec located on the eZdsp development board.

\textbf{User} \\
%The system will be able to be partially controlled by the end user. The user will be able to affect the entire system with the desired preference. The preference will be restricted to gain in the lower frequency as wished by DALI.
The user will be able to control the graphic equalizer via a GUI. 
%\todo[inline]{Reference til Samtale med DALI}

\textbf{Graphic user interface}\\
%The GUI will be based on physical hardware. The hardware will be simple of construction. The user will be browsing through the different preset using a push down button. The different preset will be displayed on the on board LCD display.
The GUI will be based on a C\# program where the user will be able to control the gain in the system's five bands.

\textbf{Graphic equalizer}\\
The five band graphic equalizer will be user controlled from the GUI. The reasoning behind implementing a five band graphic equalizer also mentioned in \autoref{sec:PreAnalysisCon}, is that the graphic equalizer is easily implemented into the system because of the four band RMS limiter. To implement the equalizer the only thing needed is to implement a variable gain block before the RMS limiter.

\textbf{RMS limiter}\\
The four octave band RMS limiter, meeting the requirements of IEC 6964 (2001), is based on a RMS detector block in each band which will detect the RMS level. From the RMS level a variable gain, gains based on a threshold of 150 W. 

This leads to the following requirement specification. 














%\textbf{Effect controller} \\
%The effect controller will be consisting of a graphical user interface in which the developer can adjust the entire frequency response according to the wanted output. The effect controller is thought as a fast prototyping function in which the developer can adjust on the fly when tuning the speaker system.

%\textbf{Developer} \\
%The developer is DALI or the specific engineer who wishes to tune and adjust the product.

% \textbf{User Equalizer} \\
% The user equalizers job is to compensate for any needed or unwanted adjustment in the system. This equalizer will only be changing whenever the user wants it or during the design period. The equalizer will consist of 4 Bi-quad bands an a high/low shelf which is to be spread equally across a frequency spectrum of 20 Hz trough 20 kHz.

% \textbf{Delay} \\
% The delay is created in order to give the monitoring system time for analysis. The delay will be adjusted accordingly to the analysis time needed in the feed forward system.


% \textbf{Peak Detection} \\
% The Peak detection block will be consisting of a signal analysis and estimation of the incoming signal. The system will be checking all amplitudes of the signal in order to determine whether a signal is to large and it should be adjusted for. The samples will be measured and compared with a predetermined threshold to determine if further action should be done.

% \textbf{Spectrum Analysis} \\
% When the signal has been analysed for potential amplitude spikes further analysis is done to estimate at what frequency these spikes occur. A spectral estimation will locate at what frequency the high amplitude is happening.  


% \textbf{Decision} \\
% When the spectrum analysis shows fluctuations above a certain threshold it should determine what attenuation is needed and send information to the Auto equalizer to adjust accordingly. The decision block will be weighting all of the signal accordingly with a model for the speaker. The model should weight the problematic frequencies higher than the less problematic frequencies. A problematic is defined as a frequency which would create to large an excursion of the diaphragm. 



% \textbf{Equalizer} \\
% This equalizer will receive filter constants released/updated by the Decision and Model block. If there is no need of adjustment it will simply adjust by a gain of 1 in the entire spectrum. The equalizer will be consisting of a parametric equalizer which will then adjust according to the information given by the decision block.

% \textbf{Limiter} \\
% The limiter is greyed out and will not be discussed further on. It is noted that there should be room in the code for implementation of a multi band limiter wished by DALI.

% \textbf{Amplification stage and Speaker driver} \\
% The speaker is in this project seen as a black box which will not be tampered with. The black box will function as an actuator which is to be controlled by the DSP system. How ever it is assumed that the DSP system will prolong the life of the speaker driver when pushed to its limits.


% \vspace{5mm}
% The overall system has now been described in detail. The different functions placed in the boxes have been outlined and it is now possible to determine more specific requirements for the system.
