\chapter{Overview} \label{ch:overview}
This chapter will describe a of the sytem before the requirements of the system is set up. A overview of the system can be seen on \autoref{fig:SystemOverview} which shows a feedback system which regulates \textbf{TBD} the volume of the sub bass based on the \textbf{TBD} model, seen on \textbf{figure TBD}, of the second and third harmonic relative to keynote. Besides the feedback system a user controlled equalizer should also be part of the system to give the user more flexibility.            

\begin{figure}[H]
\centering
\tikzsetnextfilename{SystemOverview}
\scalebox{0.8}{
\begin{tikzpicture}
%% Kasser %%
\node [block, fill=white] (AudioSource) at (0,0) {Audio Source};

%% DSP %%
\node [Twolineblock, fill=white] (Equalizer) at ($(AudioSource)+(3.5,0)$) {User Equalizer};

\node [Twolineblock, fill=white] (Decisionblock) at ($(Equalizer)+(0,-2)$) {RMS Detection};

\node [block, fill=white] (Delay) at ($(Equalizer)+(3.25,0)$) {Delay};

\node [block, fill=gray,text=white] (Limiter) at ($(Delay)+(6.5,0)$) {Limiter};

\node [Twolineblock, fill=white] (AmplitudeDetect) at ($(Delay)+(0,-2)$) {Spectrum Analysis};

\node [Twolineblock, fill=white] (Spectrum) at ($(AmplitudeDetect)+(3.25,0)$) {Decision};

\node [Twolineblock, fill=white] (NewEQ) at ($(Spectrum)+(3.25,0)$) {Equalizer};

\node [circle, fill=white] (Sum) at ($(Equalizer)+(0,-2)$) {+};

%\node [Twolineblock, fill=white] (SpectralAnalysis) at ($(Decisionblock)+(3.5,0)$) {Spectral analysis};


%% External %%
\node [block, fill=white] (UserInterface) at ($(Equalizer)+(-2,2)$) {User interface};
%\node [Twolineblock, fill=white] (EffectController) at ($(Equalizer)+(2,2)$) {Effect controller};

%% Speaker Enclosure %%
\node [Twolineblock, fill=black!80,text=white] (Amplifier) at ($(Limiter)+(4,0)$) {Amplification stage};
\node [Twolineblock, fill=black!80,text=white] (Driver) at ($(Amplifier)+(0,-2)$) {Speaker driver};

%\node [Twolineblock, fill=white] (SensorDriver) at ($(Amplifier)+(0,-2)$) {Sensor on driver};
%\node [Twolineblock, fill=gray,text=white] (EnclosureDriver) at ($(Driver)+(0,-2)$) {Sensor in enclosure};

\node (User) at ($(-2.5,0)+(UserInterface)$) {\textbf{User}};
\draw[->] (User) -- (UserInterface);

%\node (Developer) at ($(3.5,0)+(EffectController)$) {\textbf{Developer}};
%\draw[<-] (EffectController) -- (Developer);

%% Store Blokke %%
\begin{pgfonlayer}{bg}
\draw[thick, fill=black!20] ($(-1.50,1)+(Equalizer)$) -- ($(1.750,1)+(Limiter)$) |- ($(1.750,-1.25)+(Decisionblock)$) -- ($(-1.50,-1.25)+(Decisionblock)$) -- ($(-1.50,1)+(Equalizer)$);
\node (DSPtag) at ($(1,-3)+(Limiter)$) {\textbf{DSP}};

\draw[thick, fill=black!20] ($(-1.50,1)+(Amplifier)$) -| ($(1.750,1)+(Driver)$) |- ($(-1.50,-1.25)+(Driver)$) -| ($(-1.50,1)+(Amplifier)$);
\node (DSPtag) at ($(0,-1)+(Driver)$) {\scalebox{0.7}{\textbf{Speaker Enclosure}}};
\end{pgfonlayer}

%% Forbindelse %%
\draw[->] (AudioSource) -- (Equalizer);
\draw[->] (Equalizer) -- (Delay);
\draw[->] (Delay) -- (Limiter);
\draw[->] (NewEQ) -- (Limiter);

\draw[->] (Limiter) -- (Amplifier);
\draw[->] (Amplifier) -- (Driver);

\draw[->] (UserInterface) -| ($(Equalizer)+(-.25,.6)$);
%\draw[->] (EffectController) -| ($(Equalizer)+(.25,.6)$);

%\draw[->] ($(SensorDriver)+(-1.375,-0.3)$) -- ($(SpectralAnalysis)+(1.375,-0.3)$);

%\draw[->] ($(EnclosureDriver)+(-1.375,0.3)$) -- ($(EnclosureDriver)+(-1.75,0.3)$) -- ($(EnclosureDriver)+(-1.75,1)$) -- ($(EnclosureDriver)+(-5.5,1)$) |- ($(SpectralAnalysis)+(1.375,0.3)$);

%\draw[->] (SpectralAnalysis) -- (Decisionblock);
\draw[<-] (Decisionblock) -- (Equalizer);
\draw[->] (Decisionblock) -- (AmplitudeDetect);
\draw[->] (AmplitudeDetect) -- (Spectrum);
\draw[->] (Spectrum) -- (NewEQ);
\end{tikzpicture}}
\caption{Overview of the system}
\label{fig:SystemOverview}
\end{figure}

A more detailed description of every block will now follow.

\textbf{Audio source} \\
The audio source will in this project be an analog line input. 

\textbf{User} \\
The user of the system will both be an commercial user and a developer as described in \autoref{sec:user_interface}. 

\textbf{User controlled equalizer}\\
The user controlled equalizer will be an equalizer which can be controlled by different users for different purposes.

\textbf{User interface} \\
The interface is t 

\textbf{Sensor} \\
The sensor is the accelerometer located on the driver as described in \autoref{sec:Sensor}, which measures the vibration of the driver to give the signal analysis block data to analyze. 

\textbf{Signal analysis} \\
The signal analysis block is the part of the system which makes the signal analysis, this could for example be a FFT or anything else dependent on what is nescesarry for the decision block.

\textbf{Decision block} \\
The decision block decides how the regulating equalizer should regulate the input signal based on the input from the signal analysis block. 

\textbf{Regulating equalizer} \\
The regulating equalizer 

\textbf{Limiter} \\
The limiter

\textbf{Amplification stage} \\
The amplification stage

\textbf{Speaker driver} \\
The speaker driver

